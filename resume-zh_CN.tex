% !TEX TS-program = xelatex
% !TEX encoding = UTF-8 Unicode
% !Mode:: "TeX:UTF-8"

\documentclass{resume}
\usepackage{zh_CN-Adobefonts_external} % Simplified Chinese Support using external fonts (./fonts/zh_CN-Adobe/)
%\usepackage{zh_CN-Adobefonts_internal} % Simplified Chinese Support using system fonts
\usepackage{linespacing_fix} % disable extra space before next section
\usepackage{cite}

\begin{document}
\pagenumbering{gobble} % suppress displaying page number

\name{姓名}

% {E-mail}{mobilephone}{homepage}
% be careful of _ in emaill address
\contactInfo{(+86) 123-4567-8901}{xxx@xxx.com}{Web前端研发工程师}{GitHub @hijiangtao}
% {E-mail}{mobilephone}
% keep the last empty braces!
%\contactInfo{xxx@yuanbin.me}{(+86) 131-221-87xxx}{}
 
\section{个人总结}
本人在校成绩优秀、乐观向上,工作负责、自我驱动力强、热爱尝试新事物,认同开放、连接、共享的Web在未来的不可替代性。在校期间长期从事可视分析(Web的2D/3D时空可视化)相关研究,对Web技术发展趋势及前端工程化解决方案有浓厚兴趣。\textbf{现任职于 BAT 集团。}

% \section{\faGraduationCap\ 教育背景}
\section{教育背景}
\datedsubsection{\textbf{中国科学院大学},计算机应用技术,\textit{在读硕士研究生}}{2015.9 - 2018.6}
\ \textbf{排名11/133(前10\%)},中国科学院大学学业奖学金(2次),IEEE Student member,预计2018年6月毕业
\datedsubsection{\textbf{北京理工大学},软件工程,\textit{工学学士}}{2011.9 - 2015.6}
\ \textbf{排名2/62(前5\%)},国家励志奖学金,人民奖学金(7次),科技竞赛奖(2次),北京市普通高等学校优秀毕业生,北京理工大学优秀毕业生,软件学院金牌毕业生,优秀团员/优秀学生(5次)
\datedsubsection{\textbf{荷兰 莱顿大学},计算机科学与技术,\textit{国家留学基金委公派交换生}}{2015.3 - 2015.5}
\ 2014年中国政府奖学金(\textit{http://www.csc.edu.cn/}),DID-ACTE项目交换生(\textit{http://did-acte.org/})

% \section{\faCogs\ IT 技能}
\section{技术能力}
% increase linespacing [parsep=0.5ex]
\begin{itemize}[parsep=0.2ex]
  \item \textbf{编程语言}: JavaScript (ECMAScript, Node.js), HTML/CSS, Python, Go, SQL, C, Shell
  \item \textbf{操作系统,数据库与工程构建}: Linux/macOS/MySQL/MongoDB/Git/webpack/Progressive Web App
  \item \textbf{关键词}: React/Vue.js/D3.js(SVG)/three.js(canvas, WebGL)/chrome extension/Express
\end{itemize}

% \end{itemize}

\section{实习经历}
\datedsubsection{\textbf{XXXX集团 | XXXXX}, 前端开发工程师}{2011-2017}
\begin{itemize}
  \item 北京前端团队全面负责 web 应用与基础架构框架研发
  \item 独立负责 XX 需求(React),通过HTML5 本地存储及JSBridge实现在XXX发布上线
  \item 独立负责 chrome 插件开发,完成 XXX 等页面的开发与交叉营销的接入工作
\end{itemize}

\datedsubsection{\textbf{YYYY科技有限公司 | YYYYYY},数据挖掘与可视化工程师}{2005-2011}
\begin{itemize}
  \item \textbf{利用海量用户定位数据,对城市空间及人群移动特征进行研究。}第一个课题是基于香农熵和人群出行模式,构建城市网格与用户矩阵分析城市多样性/流动性分布;可视分析平台前端与可视化基于D3/Vue/Express开发,数据分析与存储采用Python/MySQL/MongoDB技术,为了均衡大数据情况下的页面可视化渲染消耗用canvas替代svg。第二个课题是对海量商场定位数据做人群分类与可视化查询,依据该系统撰写的论文被CIKM 2016(DAVA Workshop)录用,并收录于中科院软件所年会成果集
  \item 负责数据科学部HQ LAB的可视化原型开发,主导 TalkingMind 平台系统设计与前端开发
\end{itemize}

\datedsubsection{\textbf{北京ZZZZ信息技术有限公司 | ZZZZZZ},Web开发工程师}{2005-2005}
\begin{itemize}
  \item \textbf{独立负责MUSE部门的可视化组件研发。}与平台研发、设计协作完成 DeepGlint Developer 平台可视化图表组件的集成开发,符合完全定制化渲染、响应式布局与实时更新等特点
  \item 利用 D3+Vue+WebGL(Three.js) 尝试实现三维空间的人群移动可视化
\end{itemize}

% \begin{onehalfspacing}
% \end{onehalfspacing}

% \datedsubsection{\textbf{DID-ACTE} 荷兰莱顿}{2015年}
% \role{本科毕业设计}{LIACS 交换生}
% 利用结巴分词对中国古文进行分词与词性标注,用已有领域知识训练形成 classifier 并对结果进行调优
% \begin{onehalfspacing}
% \begin{itemize}
%   \item 利用结巴分词对中国古文进行分词与词性标注
%   \item 利用已有领域知识训练形成 classifier, 并用分词结果进行测试反馈
%   \item 尝试不同规则,对 classifier 进行调优
% \end{itemize}
% \end{onehalfspacing}

\section{竞赛获奖/项目作品}
% increase linespacing [parsep=0.5ex]
\begin{itemize}[parsep=0.2ex]
%   \item LeetCodeOJ Solutions, \textit{https://github.com/hijiangtao/LeetCodeOJ}
  \item 第三届中国软件杯大学生软件设计大赛\textbf{全国一等奖}( \textit{http://www.cnsoftbei.com/} ),2014 年8月
  \item 中国机器人大赛创意设计大赛\textbf{全国特等奖}( \textit{http://www.rcccaa.org/} ),2013年8月
%   \item 中国机器人大赛暨Robocup公开赛(武术擂台赛)全国一等奖,2013年10月
  \item 第11届北京理工大学“世纪杯”竞赛学生课外科技作品竞赛\textbf{特等奖},2013年8月
  \item VIS Components for security system, \textit{https://hijiangtao.github.io/ss-vis-component/}
  \item 个人博客:\textit{https://hijiangtao.github.io/},更多作品见 \textit{https://github.com/hijiangtao}
%   \item 电视节目"爸爸去哪儿"可视化分析展示, \textit{https://hijiangtao.github.io/variety-show-hot-spot-vis/}
\end{itemize}

% \section{\faHeartO\ 项目/作品摘要}
% \section{项目/作品摘要}
% \datedline{\textit{An Integrated Version of Security Monitor Vis System}, https://hijiangtao.github.io/ss-vis-component/ }{}
% \datedline{\textit{Dark-Tech}, https://github.com/hijiangtao/dark-tech/ }{}
% \datedline{\textit{融合社交网络数据挖掘的电视节目可视化分析系统}, https://hijiangtao.github.io/variety-show-hot-spot-vis/}{}
% \datedline{\textit{LeetCodeOJ Solutions}, https://github.com/hijiangtao/LeetCodeOJ}{}
% \datedline{\textit{Info-Vis}, https://github.com/ISCAS-VIS/infovis-ucas}{}


% \section{\faInfo\ 社会实践/其他}
\section{社区参与/实践其他}
% increase linespacing [parsep=0.5ex]
\begin{itemize}[parsep=0.2ex]
  \item 乐于参与开源社区讨论,\textbf{参与翻译 Vue.js, webpack, WebAssembly, Babel 文档,印记中文成员}
  \item 中国科学院大学2016秋季学期可视化与可视分析课程助教,\textit{http://vis.ios.ac.cn/infovis-ucas/}
  \item 未来论坛学生会成员、北理社联新闻信息中心主任、北理工软件学院学生会宣传部副部长(2012-2016)
  \item 2013-2015 北京市共青团“温暖衣冬”志愿者,第九届园博会志愿者,2014 FLL机器人世锦赛志愿者
\end{itemize}

%% Reference
%\newpage
%\bibliographystyle{IEEETran}
%\bibliography{mycite}
\end{document}
